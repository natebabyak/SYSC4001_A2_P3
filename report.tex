\newcommand{\thetitle}{SYSC 4001 Assignment 2 --- Part III}

\documentclass[12pt,titlepage]{article}
\usepackage{fancyhdr}
\usepackage[headheight=15pt,margin=1in]{geometry}
\usepackage{lastpage}
\usepackage[allcolors=blue,colorlinks=true]{hyperref}

\pagestyle{fancy}
\fancyhead{}
\fancyhead[R]{\thetitle}
\fancyfoot{}
\fancyfoot[C]{Page \thepage\ of\ \pageref{LastPage}}

\title{\bfseries\thetitle}
\author{Nate Babyak --- 101310590 \\ Ozan Kaya --- 101322055}
\date{November 7, 2025}

\begin{document}

\maketitle

\section{Introduction}

The objective of this part of the assignment is to build a small API simulator. This simulator has one CPU, fixed partitions in memory, and is primarily focused at providing functionality for \verb|fork()| and \verb|exec()|. This final part of the assignment builds on the final part of the previous assignment, where we simulated an interrupt system.

\section{Tests}

\subsection{Test 1}

\subsubsection{Analysis of \href{https://github.com/natebabyak/SYSC4001_A2_P3/blob/main/output_files/test_1/execution.txt}{\texttt{execution.txt}}}

Init forks a child process. The child executes \verb|program1| using exec, which is loaded into a 10 Mb partition, and its PCB is updated. Upon child completion, the parent will execute \verb|program2| in a separate 15 Mb partition. Each system call causes an ISR to be triggered, which will be followed by memory partition allocation and a scheduler call. The execution log contains timestamps for context switches, entry into kernel mode, and CPU bursts to demonstrate that fork/exec and priority handling are being sequenced correctly.

\subsubsection{Analysis of \href{https://github.com/natebabyak/SYSC4001_A2_P3/blob/main/output_files/test_1/system_status.txt}{\texttt{system\_status.txt}}}

At time 24, the system executes \verb|FORK, 10| creating a child process. The child (PID 1) inherits the parent's PCB and is made to run, while the parent (PID 0) is made to wait because children have a higher priority. Around time 256, the child executes \verb|EXEC program1, 50|. Using best-fit allocation, \verb|program1| (10 Mb) is allocated partition number 4 (10 Mb). Around time 633, after the child completes execution, the parent executes \verb|EXEC program2, 25|, which allocates partition number 3 (15 Mb) to \verb|program2| (15 Mb).

\subsection{Test 2}

\subsubsection{Analysis of \href{https://github.com/natebabyak/SYSC4001_A2_P3/blob/main/output_files/test_2/execution.txt}{\texttt{execution.txt}}}

Init forks a child process. The child executes \verb|program1| using \verb|EXEC|. This program forks a child process, does nothing with it, then excutes \verb|program2|. After this program executes, we return to the

\subsubsection{Analysis of \href{https://github.com/natebabyak/SYSC4001_A2_P3/blob/main/output_files/test_2/system_status.txt}{\texttt{system\_status.txt}}}

Around time 31, the system executes \verb|FORK|, creating a child process. The child (PID 1) has the same PCB as its parent (PID 0) and is set to running. Around time 229, the child executes \verb|program1|, which takes 10 Mb, and is thus allocated partition number 4 (10 Mb). Around time 258, \verb|program1| executes \verb|FORK|, which creates another \verb|program1| with PID 2 and at partition number 3. Around time 543, the child executes \verb|program2|.

\subsection{Test 3}

\subsubsection{Analysis of \href{https://github.com/natebabyak/SYSC4001_A2_P3/blob/main/output_files/test_3/execution.txt}{\texttt{execution.txt}}}

In this test, \verb|FORK| is called and \verb|program1| is loaded into the parent, while child is left untouched. The child then runs until completion before the parent executes.

\subsubsection{Analysis of \href{https://github.com/natebabyak/SYSC4001_A2_P3/blob/main/output_files/test_3/system_status.txt}{\texttt{system\_status.txt}}}

At around time 34, \verb|init| executes \verb|FORK|, which creates a child (PID 1) with the same PCB as the parent (PID 0). The child is set to running and, using best-fit allocation, is allocated partition number 5. At around time 286, the child has completed and the parent executes \verb|program1|, which is 10 Mb, and is thus allocated parition number 4 (10 Mb).

\subsection{Test 4}

\subsubsection{Analysis of \href{https://github.com/natebabyak/SYSC4001_A2_P3/blob/main/output_files/test_4/execution.txt}{\texttt{execution.txt}}}

This test starts with a \verb|FORK|, which clone the current PCB into the process. Then, the memory image of \verb|program1| is copied into the child. This program also invokes \verb|FORK| and this child executes \verb|program2|. The programs run child-first, meaning \verb|program2| first, then \verb|program1|, and finally \verb|program3| is executed by \verb|init|.

\subsubsection{Analysis of \href{https://github.com/natebabyak/SYSC4001_A2_P3/blob/main/output_files/test_4/system_status.txt}{\texttt{system\_status.txt}}}

At around time 26, \verb|FORK| is invoked, resulting in a child process (PID 1). At around time 228, the child process executes \verb|program1|, which occupies parition number 4. At around time 252, \verb|program1| executes \verb|FORK| itself, creating another \verb|program1| with PID 2. This child then executes \verb|program2|, which occupies paritition number 5 as 4 is already occupied. Finally, at around time 1791, after the two children have completed, \verb|init| executes \verb|program3| (10 Mb), which is allocated parititon number 3 (15 Mb).

\subsection{Test 5}

\subsubsection{Analysis of \href{https://github.com/natebabyak/SYSC4001_A2_P3/blob/main/output_files/test_5/execution.txt}{\texttt{execution.txt}}}

First, \verb|FORK| is used to create a child (PID 1), which starts running. This child executes \verb|program1|, which then runs to completion. The parent (PID 1) then executes \verb|program2|, which also runs to completion.

\subsubsection{Analysis of \href{https://github.com/natebabyak/SYSC4001_A2_P3/blob/main/output_files/test_5/system_status.txt}{\texttt{system\_status.txt}}}

At around time 32, \verb|init| invokes \verb|FORK|, which creates a child (PID 1) and allocates it partition number 5. At around time 274, the child process exeuctes \verb|program1|, which is 12 Mb and is thus allocated parition number 3 (15 Mb). After \verb|program1| completes, the parent process executes \verb|program2|, which is allocated parition number 2 (25 Mb) to fit its 18 Mb size. Finally \verb|program2| runs until completion.

\section{Conclusion}

In conclusion,

\section*{Appendix}

\begin{itemize}
    \item \href{https://github.com/natebabyak/SYSC4001_A2_P2}{SYSC 4001 Assignment 2 --- Part II}
    \item \href{https://github.com/natebabyak/SYSC4001_A2_P3}{SYSC 4001 Assignment 2 --- Part III}
\end{itemize}

\end{document}