\newcommand{\thetitle}{SYSC 4001 Assignment 2 --- Part III}

\documentclass[12pt,titlepage]{article}
\usepackage{fancyhdr}
\usepackage[headheight=15pt,margin=1in]{geometry}
\usepackage{lastpage}
\usepackage[allcolors=blue,colorlinks=true]{hyperref}

\pagestyle{fancy}
\fancyhead{}
\fancyhead[R]{\thetitle}
\fancyfoot{}
\fancyfoot[C]{Page \thepage\ of\ \pageref{LastPage}}

\title{\bfseries\thetitle}
\author{Nate Babyak --- 101310590 \\ Ozan Kaya --- 101322055}
\date{November 7, 2025}

\begin{document}

\maketitle

\section{Introduction}

The goal of this part is to simulate \verb|fork()| and \verb|exec()| system calls using a single CPU, fixed memory partitions, PCB, and external files.

\section{Tests}

\subsection{Test 1}

\subsubsection{Analysis of \href{https://github.com/natebabyak/SYSC4001_A2_P3/blob/main/output_files/test_1/execution.txt}{\texttt{execution.txt}}}

Init forks a child process. The child executes \verb|program1| using exec, which is loaded into a 10 Mb partition, and its PCB is updated. Upon child completion, the parent will execute \verb|program2| in a separate 15 Mb partition. Each system call causes an ISR to be triggered, which will be followed by memory partition allocation and a scheduler call. The execution log contains timestamps for context switches, entry into kernel mode, and CPU bursts to demonstrate that fork/exec and priority handling are being sequenced correctly.

\subsubsection{Analysis of \href{https://github.com/natebabyak/SYSC4001_A2_P3/blob/main/output_files/test_1/system_status.txt}{\texttt{system\_status.txt}}}

best-fit allocation

\subsection{Test 2}

\subsection{Test 3}

\subsection{Test 4}

\subsection{Test 5}

\section{Conclusion}

\section*{Appendix}

\begin{itemize}
    \item \href{https://github.com/natebabyak/SYSC4001_A2_P2}{SYSC 4001 Assignment 2 --- Part II}
    \item \href{https://github.com/natebabyak/SYSC4001_A2_P2}{SYSC 4001 Assignment 2 --- Part III}
\end{itemize}

\end{document}